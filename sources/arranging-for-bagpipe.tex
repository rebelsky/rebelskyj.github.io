\documentclass[12pt]{article}[titlepage]
\newcommand{\say}[1]{``#1''}
\newcommand{\nsay}[1]{`#1'}
\usepackage{endnotes}
\newcommand{\1}{\={a}}
\newcommand{\2}{\={e}}
\newcommand{\3}{\={\i}}
\newcommand{\4}{\=o}
\newcommand{\5}{\=u}
\newcommand{\6}{\={A}}
\newcommand{\B}{\backslash{}}
\renewcommand{\,}{\textsuperscript{,}}
\usepackage{setspace}
\usepackage{hyperref}
\begin{document}
\doublespacing

\section{\href{arranging-for-bagpipe.html}{Arranging for Bagpipe}}
Prereading note: I definitely have far more footnotes here (59 in total, 34 in the final draft) than normal (between 1 and 17 in total, and between 1 and 8 in the final draft) in my blog thus far. I have no clue why I felt the need for so many notes while writing this post, but they feel pretty needed.
\section{Draft 2}
This spring, I started learning the Scottish bagpipes.\footnote{I read somewhere\footnote[2]{I think it may have been a quotation in one of the Rebelsky Family Bookclub Books} that some people recommend saying that they began playing, rather than learning, a new instrument. Their logic is that otherwise it becomes unclear where the learning stops and the playing starts. Personally, I prefer studying in that sort of context, but since that brings connotations of serious or academic purposes for learning, rather than my source of desire to learn,\footnotemark{} I still use learning, since I would say that I'm learning when I practice, and playing when I perform} \footnotetext[2]{I think it may have been a quotation in one of the Rebelsky Family Bookclub Books} \footnotetext[3]{usually the feeling of \say{Ooh Shiny!} I find myself feeling when I see a new instrument}
While learning the bagpipe, one of my cross-country running friends\footnote{I hope that the runner would agree with the term friend} asked me if I would be willing to play the (United States') National Anthem\footnote{I just realized that I have no clue whether or not the United States National Anthem needs a possessive. I feel like it should, since it is the National Anthem of the United States, but I don't think I ever see it phrased that way} at the Grinnell team's only home meet, the Les Duke Invite.
That meet is today, and (I think) just ended at the time of my writing this post.
Obviously, I was unable to be there.
But, their mentioning of the idea did spark me to try to find a bagpipe arrangement for The Star-Spangled Banner.\footnote{also, I just figured out a good way to avoid the issues of both the possessive (as above) and my tendency to refer to the song as \say{The National Anthem}, which is only accurate for a small subset of the world}\,\footnote{new problem though, do you capitalize the \say{The} in the title if it's in the middle of the sentence? I assume yes, since it's a part of the title}
Initial searches were fairly fruitless, with what versions I could find sounding nothing like the anthem I knew.\footnote{I don't link them because some people might take it as an attack, and I didn't save them, so I don't have them easily available}
I found a \href{http://www.therealviperpiper.com/viewtopic.php?t=11731}{forum} discussing this, and they concluded that the anthem is unplayable on the bagpipes.

Now, those of you who know me may know that I don't take being told something is impossible well.\footnote{some of you may be calling this an understatement}
It feels like a challenge.
Some of you may also know that I\footnote{try to} compose music.
The brilliant\footnote{read: petty} part of my mind thought that I would be able to arrange the piece for bagpipes.
Obviously,\footnote{to those of you who know that the bagpipe is a 9 note diatonic instrument starting on the subdominant in the key of D Major\footnotemark}\footnotetext{if the above words made no sense, that's ok, they're just music jargon saying the bagpipe plays a\footnotemark{} version of nine white piano keys, starting on an F\footnotemark}\footnotetext[10]{transposed}\footnotetext[11]{the key right before the three black keys next to each other} that wouldn't fit the song,\footnote{another clever way to avoid the issue of naming} which requires an octave and a fifth range,\footnote{I think}, and some chromatic notes.\footnote{I know this one. In the key of G, it requires a C natural and a C\#}
But, you can always drop and raise octaves to fit a piece in.
Additionally, my teacher mentioned offhandedly that there are chromatic fingerings for the bagpipe.
So, I worked off of a version of the piece in the key of G.\footnote{my logic being that the piece scored in D looked like it fit in the natural range of the pipe worse than the piece scored in G}

There I ran into my first problem: I have no clue how to play a C natural on a bagpipe, which was required in the second half of the piece.\footnote{hey! I did it again}
But, more importantly than that, it was centered around G,\footnote{which should have been obvious at the time, but I tend to avoid thinking big picture when I'm working out of spite} which those of you familiar with harmonics may see is dissonant to A.\footnote{the note the drones play in}\,\footnote{to the people who will point it out, I know in higher harmonics A and G are both in the same series, but the A's are a major second above and a minor seventh below the G, which isn't consonant}

When I asked a friend of mine who plays bagpipes,\footnote{I chose play intentionally, since he performs fairly regularly, and is far more experienced than me} he told me that, in his experience, people would play the second half of the song,\footnote{starting at \say{and the rocket's red glare}} because the first half isn't doable.
That's when I suddenly realized that there's a modulation in the middle of The Star Spangled Banner.\footnote{no, the use of C natural instead of C\# and the fact that the piece is centered on D in the first half and G in the second didn't occur to me. For why, please see footnote 21}
With that in mind, I transposed the end of the piece to D major, which worked pretty well.\footnote{there were only a few notes that didn't fit well, and they're minor notes that I didn't notice the change for, probably because so many people use them as a place to improvise a little}
I then put the initial half back in the song.\footnote{which had earlier required a non-insignificant amount of octave switching, but not an undue amount}
I expected some cognitive dissonance from the\footnote{lack of} modulation in my version of the piece, but it was fairly minor.
And, since the piece was centered around D, it fit in nicely with the harmonics of the instrument.

Then came to the part of arranging for bagpipe that I struggle with: adding the embellishments.
The bagpipe plays a continuous note, so to break up repeated notes,\footnote{or add spice between different notes} small chirpy notes are played to break the sound.
There are many kinds of embellishments, including grace notes,\footnote{which are any of D, E, F only if preceding an E, and high G}, doublings and half doublings,\footnote{a high G grace note before the pitch being played, then the lowest grace note higher than the pitch being played, or the aformentioned without the high G, respectively} and many others.
As you might expect from a musical tradition lasting centuries, there's a lot of theory behind deciding what embellishments go where\footnote{as far as I can tell from being told that in nearly those exact words} which I don't know and can't easily find only.
So, I put in embellishments that I knew\footnote{and felt comfortable playing} and looked right where they were, then tried playing through.
Nothing looked or sounded horrible to me, but I also don't know enough to know if anything I do would be horrible.
So, now I have a version of the National Anthem of the United States of America that fits entirely on the bagpipe.
If anyone wants it, feel free to drop me a message at \href{mailto:flyingrebelpipes@gmail.com}{flyingrebelpipes@gmail.com}.
My only request if you end up using the piece is that you send me a recording of yourself playing it.

\section{Draft 1}
This spring, I started learning the bagpipe.\footnote{I read somewhere\footnotemark{} that some people say playing, not learning a new instrument, because otherwise it becomes unclear where the learning stops and the playing starts. Personally, I prefer studying in that sort of context, but since that brings connotations of serious or academic purposes for learning, rather than my goals,\footnotemark{} I still use learning, since I would say that I'm learning when I practice, and playing when I perform}
\footnotetext[36]{I think it may have been a quotation in one of the Rebelsky Family Bookclub Books}
\footnotetext[37]{usually less of a goal and more of a \say{Ooh Shiny!} feeling inside when I see a new instrument}
While learning the bagpipe, one of my cross-country running friends\footnote{I hope that the runner would agree with the term friend} asked me if I would be willing to play the (United States') National Anthem\footnote{I just realized that I have no clue whether or not the United States National Anthem needs a possessive. I feel like it should, since it is the National Anthem of the United States, but I don't think I ever see it phrased that way} at the Grinnell team's only home meet, the Les Duke Invite.
That meet is today, and (I think) just ended at the time of my writing this post.
Obviously, I was unable to be there.
But, their mentioning of the idea did spark me to try to find a bagpipe arrangement for The Star-Spangled Banner.\footnote{also, I just figured out a good way to avoid the issues of both the possessive (as above) and my tendency to refer to the song as \say{The National Anthem}, which is only accurate for a small subset of the world}
Initial searches were fairly fruitless, with what versions I could find sounding nothing like the anthem I knew.
I found a \href{http://www.therealviperpiper.com/viewtopic.php?t=11731}{forum} discussing this, and they concluded that the anthem is unplayable on the bagpipes.

Now, those of you who know me may know that I hate being told that something is impossible.
It feels like a challenge.
Some of you may also know that I\footnote{try to} compose music.
I thought I would be able to arrange the piece for bagpipes.
Obviously,\footnote{to those of you who know that the bagpipe is a 9 note diatonic instrument starting on the subdominant tuned to D Major\footnotemark}
\footnotetext{if the above words made no sense, it's ok, they're effectively just jargon saying the bagpipe is a\footnotemark{} version of nine white piano keys, starting on an F\footnotemark}
\footnotetext[44]{transposed}
\footnotetext[45]{the key right before the three black keys next to each other} that wouldn't fit the song,\footnote{another clever way to avoid the issue} which requires an octave and a fifth range,\footnote{I think}, and some chromatic notes.\footnote{I know. In the key of G, it requires a C natural and a C\#}
But, you can always drop and raise octaves.
So, I worked off of a version of the piece in the key of G.

There I ran into my first problem: I have no clue how to play a C natural on a bagpipe.
But, more importantly than that, it was centered around G,\footnote{which should have been obvious at the time, but I tend to avoid thinking big picture when I'm doing things out of spite} which those of you familiar with harmonics may see is dissonant to A.\footnote{the note the drones play in}\,\footnote{yes, I know in higher harmonics they may eventually be consonant, but the A's are a major second above and a minor seventh below the G, which isn't consonant}
When I asked a friend of mine who plays bagpipes,\footnote{I chose play intentionally, since he performs fairly regularly, and is far more experienced than me} he told me that, in his experience, people would play the second half of the song,\footnote{starting at \say{and the rocket's red glare}} because the first half isn't doable.
That's when I suddenly realized that there's a modulation in the middle of The Star Spangled Banner.\footnote{new problem, do you capitalize the \say{The} in the title if it's in the middle of the sentence? I assume yes, since it's a part of the title}
With that in mind, I transposed the end of the piece to D major, which worked pretty well.\footnote{there were only a few notes that didn't fit well, and they're minor notes that I didn't notice the change for, probably because so many people use them as a place to improvise a little}
I then put the initial half back in the song.\footnote{that required a fair amount of octave switching, but not an undue amount}
I expected some cognitive dissonance from the\footnote{lack of} modulation in my version of the piece, but it was fairly minor.
And, since the piece was centered around D, it fit in nicely with the harmonics of the instrument.

Then came to the part of arranging for bagpipe that is scary to me: adding the embellishments.
The bagpipe plays a continuous note, so to break up repeated notes,\footnote{or add spice between different notes} small chirpy notes are played to break the sound.
There's a lot of theory behind where what embellishments go where\footnote{as far as I can tell from being told that in nearly those exact words} which I have no clue about.
So, I put in embellishments that I knew and looked right where they were, then tried playing through.
Nothing felt horrible to me, but I also don't know enough to know if anything I do would be horrible.
So, now I have a version of the National Anthem of the United States of America that fits entirely on the bagpipe.
If anyone wants it, feel free to drop me a message at \href{mailto:flyingrebelpipes@gmail.com}{flyingrebelpipes@gmail.com}.
My only request if you end up using the piece is that you send me a recording of yourself playing it.
\end{document}