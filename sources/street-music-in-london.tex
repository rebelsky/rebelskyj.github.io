\hypertarget{street-music-in-london.html}{%
\section{Street Music in~London}\label{street-music-inlondon}}

As with most cities, London has its share of musicians performing on the
street. Although I've only been here a few days and in a small portion
of the city, I've noticed two things about the street music that I don't
like.

The first, and pettiest, of these is that I haven't seen or heard any
bagpipes. Maybe I'm just conditioned from growing up in Grinnell, where
our local bagpipe troupe practices with some regularity, but it's
shocking to me that a small Iowa town is more bagpipe-centric than
London.

Second, almost all of the musicians I've seen have been using
amplifiers. It's not that I'm an audio purist, who has a problem with
the idea of amplifying music. My problem is mostly just the fact that it
reduces the number of musicians who can play in an area. When musicians
are playing acoustic instruments \footnote{With the slight exception of
  bagpipes}, the sound is easily overwhelmed if you're more than a few
score feet away. In places like New Orleans, there's a different
musician at almost every intersection, if not also one in the middle of
the block. In London, however, the sounds are amplified, which means
that the entirety of King's Cross Station only seems to have room for
one person.

With that said, I still love the fact that there are people who are
making music near where I am. It's great, and so far they've all been
incredibly talented.

Edit 1 September: Today I saw bagpipers. It looks like there is a
rotating group who stands between the underground station near
Parliament and Parliament.
